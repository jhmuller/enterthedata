\documentclass{beamer}
\usepackage{amsmath}
\usepackage{graphicx}
\graphicspath{ {./images/} } % Declare image search paths


% Theme and color (optional)
\usetheme{Madrid}
%\usetheme{Berlin}
\usecolortheme{default}

\title{AI and Data Science Consulting}
\author{John H. Muller}
% \date{\today}

% Begin section headings
\newcommand{\intro}{Introduction}
\newcommand{\businessProblems}{Business Problems and Solutions}
\newcommand{\classify}{Classification}
\newcommand{\forecast}{Forecasting and Regression}
\newcommand{\supervised}{Supervised Learning: given labeled examples}
\newcommand{\recommend}{Recommendations}
\newcommand{\genai}{Generative AI}
\newcommand{\gen}{Generating Text}
\newcommand{\summary}{Summarizing Text}
\newcommand{\conclusion}{Conclusion and next steps}

\begin{document}

\setbeamertemplate{caption}[numbered]
\numberwithin{figure}{section}
\setcounter{figure}{0}


%----------------
% Title page
\begin{frame}
  \titlepage
  \begin{center}
    \includegraphics[width=0.5\textwidth,     height=1.5in]{enter1.jpg} %
	\end{center}
\end{frame}

% outline page
\begin{frame}{Outline}
    \tableofcontents
\end{frame}

%----------------
\section{\intro}
\subsection{About Me}
\begin{frame}
\frametitle{\intro}
\framesubtitle{About Me}
\begin{center}
    \includegraphics[width=0.5in, height=0.5in]{enter1.jpg} %
\end{center}

  \begin{itemize}
    \item Computer Scientist (PhD), data scientist and finance quant,
    \item Over 15 years of experience in industry
    \begin{itemize}
    \item Financial Services
    \item Quantitative Investing
    \item Retail
    \item Image and video compression
    \end{itemize}
    \item Passion for making my work matter.
  \end{itemize}
\end{frame}
%----------------
\subsection{Skills, Experience, and Tools}
\begin{frame}
\frametitle{\intro}
\framesubtitle{Skills, Experience and Tools}

  \begin{itemize}
  \item Computer Science: algorithms and data structures
    \item Data Science 
      \begin{itemize}
    		\item Supervised learning, clustering, regression, tree based 				methods

       \item Neural Networks, Deep Learning and Generative AI
      \end{itemize}
    \item Coding: Python and the python ecosystem
       \begin{itemize}
          \item pandas, numpy, scikit-learn, pytorch ...
         \end{itemize}
    \item SQL
    \item LangChain, Langgraph, CrewAI, Hugging Face 
    
     \end{itemize}
\end{frame}

%----------------

\section{\businessProblems}
\subsection{\classify}
\begin{frame}
  \frametitle{\businessProblems}
  \framesubtitle{Classification}
  What: Recognize "depression related" posts on Reddit.
  
  How: Use a NN to \emph{embed} posts and classify based on distance to prototypical embeddings.  
  Similar to \emph{Nearest Neighbor} classification.

   \begin{figure}[h!]
    \begin{center}
    \includegraphics[width=3in, height=1in]{textExamples.png} %
	\end{center}
	\caption{Example text from training and validation sets}
	\label{fig:textExample}
	\end{figure}

\end{frame}

%----------------
\begin{frame}
  \frametitle{\businessProblems}
  \framesubtitle{\classify}
 Examples showing embeddings in 2-D

	   \begin{figure}[h!]
	  \begin{center}
    \includegraphics[width=3in, height=.8in]{textClassifyModel.png} %
	\end{center}
		\caption{Four layer NN Model}
		\label{fig:NNModel}
	\end{figure}
\end{frame}
%----------------
\begin{frame}
  \frametitle{\businessProblems}
  \framesubtitle{\classify}
 

	   \begin{figure}[h!]
	  \begin{center}
    \includegraphics[width=3in, height=.8in]{textClassifyModel.png} %
	\end{center}
		\caption{Four layer NN Model}
		\label{fig:NNModel}
	\end{figure}
\end{frame}


%----------------
\begin{frame}
  \frametitle{\businessProblems}
  \framesubtitle{\classify}
 
How well does it work?  Is it \textbf{useful}

 
Note that the bar for \textbf{useful}?
is dependent on the application.
E.g. nuclear power vs customer recommendations.
give accuracy or error numbers.
\end{frame}

%----------------
\begin{frame}
  \frametitle{\businessProblems}
  \framesubtitle{\classify}
  What: Classify images, e.g. recognize the "type" of image. 
 \vspace{0.1in}
  
 	   \begin{figure}[h!]
	  \begin{center}
    \includegraphics[width=2.5in, height=2.5in]{image_classify_1.png} %
	\end{center}
		\caption{Daisy or Sunflower}
		\label{fig}
	\end{figure}
\end{frame}
%----------------
\begin{frame}
  \frametitle{\businessProblems}
  \framesubtitle{\classify}
  
  How: multi-layer convolutional NN.
  
 	   \begin{figure}[h!]	
		  \begin{center}
    \includegraphics[width=3in, height=2in]{image_classify_2.png} %
	\end{center}
		\caption{Keras model}
		\label{fig:the model}
	\end{figure}
\end{frame}
%----------------
\begin{frame}
  \frametitle{\businessProblems}
  \framesubtitle{\classify}
How well does it work?  Is it \textbf{useful}?

Give example of accuracy.
Also, maybe show some example misclassification
\end{frame}


%----------------

\subsection{\forecast}
\begin{frame}
\frametitle{\businessProblems}
\framesubtitle{\forecast}
What: forecast the monthly \textbf{change} in non-farm payroll.


Data Source: BLS via St. Louis Fed site FRED. 
 	   \begin{figure}[h!]
	  \begin{center}
    \includegraphics[width=4in, height=2in]{payemsEasy.png} %
	\end{center}
		\caption{NonFarm Payroll level, 2010 - 2020}
		\label{fig:payemsEasy}
	\end{figure}
\end{frame}



%----------------
\begin{frame}
\frametitle{\businessProblems}
\framesubtitle{\forecast}
Not always a steady increase.
 	   \begin{figure}[h!]
	  \begin{center}
    \includegraphics[width=4in, height=2in]{payemsHard.png} %
	\end{center}
		\caption{NonFarm Payroll level, 2008 - 2023}		\label{fig:payemsHard}
	\end{figure}
\end{frame}
%----------------
\begin{frame}
\frametitle{\businessProblems}
\framesubtitle{\forecast}
Changes in the level look less predictable.

 	   \begin{figure}[h!]
	  \begin{center}
    \includegraphics[width=4in, height=2in]{payemsChange.png} %
	\end{center}
		\caption{NonFarm Payroll, changes}		\label{fig:payemsHard}
	\end{figure}
\end{frame}
%----------------
\begin{frame}
\frametitle{\businessProblems}
\framesubtitle{\forecast}

What: forecast the next month's change in non-farm payroll.

How: regress the monthly change on 
\begin{itemize}
	\item One or more lagged values.
	\item  Unemployment rate.
	\item Civilian Labor force.
	\item Other data that might be helpful:
		\begin{itemize}
		\item GDP (quarterly)
		\item unemployment claims (weekly)
		\item ADP payroll changes (monthly)
		\item Google trend values for "job", "layoff", ... (hourly)
		\end{itemize}
\end{itemize}
\end{frame}

%----------------
\begin{frame}
  \frametitle{\businessProblems}
  \framesubtitle{\forecast}
How well does it work?  Is it \textbf{useful}?

Useful if it can reliably predict whether the number will beat expectations.

show plot of actual vs. predicted
\end{frame}


%----------------

\subsection{\recommend}
\begin{frame}
  \frametitle{\businessProblems}
  \framesubtitle{\recommend}
  What: predict a viewer's rating for a movie.
  
  How: Embedding NN for both viewer and the movie.
  
  The similarity between viewer and rating determines the predicted rating.
  
\end{frame}

%----------------

\begin{frame}
  \frametitle{\businessProblems}
  \framesubtitle{\recommend}
Simple explanation of the model.

Back to the plot explaining embedding.
  
\end{frame}


%----------------

\subsection{\genai}
\subsubsection{\gen}
\begin{frame}
\frametitle{\businessProblems}
  \framesubtitle{\genai}
  
    What:  Given a minimal set of \emph{facts}, 
  generate a narrative or story about the facts.
  
\end{frame}

%----------------

\subsubsection{\summary}
\begin{frame}
\frametitle{\businessProblems}
  \framesubtitle{\genai}
  
  What:  Given a minimal set of \emph{facts}, 
  generate a narrative or story about the facts.
\end{frame}

%----------------

\section{\conclusion}
\begin{frame}
  \frametitle{\conclusion}

Let me help you solve your problems.

Next steps?
\end{frame}

%----------------

\section{\conclusion}
\begin{frame}
  \frametitle{\conclusion}
  \begin{center}
  Thank you!\\
\vspace{0.3in}  
  John H. Muller\\
  \vspace{0.2in}  
  jmuller.ics88@gtalumni.org \\
\vspace{0.2in}    
  +1 (617) 669-2204
  \end{center}
\end{frame}


\end{document}

