\documentclass{beamer}
\usepackage{amsmath}
\usepackage{graphicx}
\graphicspath{ {./images/} } % Declare image search paths

\usepackage{listings}
\usepackage{xcolor} % For colors

\lstset{
    language=Python,
    backgroundcolor=\color{lightgray!20},
    commentstyle=\color{green!50!black},
    keywordstyle=\color{blue},
    stringstyle=\color{red!70!black},
    showstringspaces=false,
    numberstyle=\tiny\color{gray},
    numbers=left,
    breaklines=true,
    tabsize=4,
}

% Theme and color (optional)
\usetheme{Madrid}
%\usetheme{Berlin}
\usecolortheme{default}

\title{AI and Data Science Consulting}
\author{John H. Muller}
% \date{\today}

% Begin section headings
\newcommand{\intro}{Introduction}
\newcommand{\businessProblems}{Business Problems and Solutions}
\newcommand{\classify}{Classification}
\newcommand{\forecast}{Forecasting and Regression}
\newcommand{\supervised}{Supervised Learning: given labeled examples}
\newcommand{\recommend}{Recommendations}
\newcommand{\genai}{Generative AI}
\newcommand{\gen}{Generating Text}
\newcommand{\summary}{Summarizing Text}
\newcommand{\conclusion}{Conclusion and next steps}

\begin{document}

\setbeamertemplate{caption}[numbered]
\numberwithin{figure}{section}
\setcounter{figure}{0}


%----------------
% Title page
\begin{frame}
  \titlepage
  \begin{center}
    \includegraphics[width=0.5\textwidth,     height=1.5in]{enter1.jpg} %
	\end{center}
\end{frame}

% outline page
\begin{frame}{Outline}
    \tableofcontents
\end{frame}

%----------------
\section{\intro}
\subsection{About Me}
\begin{frame}
\frametitle{\intro}
\framesubtitle{About Me}
\begin{center}
    \includegraphics[width=0.5in, height=0.5in]{enter1.jpg} %
\end{center}

  \begin{itemize}
    \item Computer Scientist (PhD), data scientist and finance quant,
    \item Over 15 years of experience in industry
    \begin{itemize}
    \item Financial Services
    \item Quantitative Investing
    \item Retail
    \item Image and video compression
    \end{itemize}
    \item Passion for making my work matter.
  \end{itemize}
\end{frame}
%----------------
\subsection{Skills, Experience, and Tools}
\begin{frame}
\frametitle{\intro}
\framesubtitle{Skills, Experience and Tools}

  \begin{itemize}
  \item Computer Science: algorithms and data structures
    \item Data Science 
      \begin{itemize}
    		\item Supervised learning, clustering, regression, tree based 				methods

       \item Neural Networks, Deep Learning and Generative AI
      \end{itemize}
    \item Coding: Python and the python ecosystem
       \begin{itemize}
          \item pandas, numpy, scikit-learn, pytorch ...
         \end{itemize}
    \item SQL
    \item LangChain, Langgraph, CrewAI, Hugging Face 
    
     \end{itemize}
\end{frame}

%----------------

\section{\businessProblems}
\subsection{\classify}
\begin{frame}
  \frametitle{\businessProblems}
  \framesubtitle{\classify}
\textbf{What}: Compare 3 methods of classifying 
on a medical dataset.

For details on the dataset see Kaggle
https://www.kaggle.com/datasets/fedesoriano/stroke-prediction-dataset


\end{frame}

\begin{frame}
  \frametitle{\businessProblems}
  \framesubtitle{\classify}
Variables
{\scriptsize
\begin{enumerate}
\item id: unique identifier
\item gender: "Male", "Female" or "Other"
\item  age: age of the patient
\item  hypertension: 0 if the patient doesn't have hypertension, 1 if the patient has hypertension
\item heart\_disease: 0 if the patient doesn't have any heart diseases, 1 if the patient has a heart disease
\item ever\_married: "No" or "Yes"
\item work\_type: "children", "Govt\_jov", "Never\_worked", "Private" or "Self-employed"
\item Residence\_type: "Rural" or "Urban"
\item  avg\_glucose\_level: average glucose level in blood
\item  bmi: body mass index
\item smoking\_status: "formerly smoked", "never smoked", "smokes" or "Unknown"*
\item stroke: 1 if the patient had a stroke or 0 if not
\end{enumerate} }
\end{frame}

%----------------
\begin{frame}
  \frametitle{\businessProblems}
  \framesubtitle{\classify}
\begin{itemize}
\item logistic regression
\item gradient boosting
\item random forests
\end{itemize}
\end{frame}
%----------------

%----------------
\begin{frame}
  \frametitle{\businessProblems}
  \framesubtitle{\classify}
 
Accuracy and loss on train, validation and test datasets. \
Loss is binary cross-entropy loss.

 	   \begin{figure}[h!]
	  \begin{center}
    \includegraphics[width=2.5in, height=.8in]{logistic.png} %
	\end{center}
		\caption{Logistic Regression}
		\label{fig:logisic}
	\end{figure}


	 	   \begin{figure}[h!]
	  \begin{center}
    \includegraphics[width=2.5in, height=.8in]{randomForest.png} %
	\end{center}
		\caption{RandomForest}
		\label{fig:randomForest}
	\end{figure}

\end{frame}


\begin{frame}
  \frametitle{\businessProblems}
  \framesubtitle{\classify}
		 	   \begin{figure}[h!]
	  \begin{center}
    \includegraphics[width=2.5in, height=.8in]{gradientBoosting.png} %
	\end{center}
		\caption{Gradient Boosting}
		\label{fig:logisic}
	\end{figure}

\end{frame}
%----------------




%----------------

\subsection{\forecast}
\begin{frame}
\frametitle{\businessProblems}
\framesubtitle{\forecast}
What: forecast the monthly \textbf{change} in non-farm payroll.


Data Source: BLS via St. Louis Fed site FRED. 
 	   \begin{figure}[h!]
	  \begin{center}
    \includegraphics[width=4in, height=2in]{payemsEasy.png} %
	\end{center}
		\caption{NonFarm Payroll level, 2010 - 2020}
		\label{fig:payemsEasy}
	\end{figure}
\end{frame}



%----------------
\begin{frame}
\frametitle{\businessProblems}
\framesubtitle{\forecast}
Not always a steady increase.
 	   \begin{figure}[h!]
	  \begin{center}
    \includegraphics[width=4in, height=2in]{payemsHard.png} %
	\end{center}
		\caption{NonFarm Payroll level, 2008 - 2023}		\label{fig:payemsHard}
	\end{figure}
\end{frame}
%----------------
\begin{frame}
\frametitle{\businessProblems}
\framesubtitle{\forecast}
Changes in the level look less predictable.

 	   \begin{figure}[h!]
	  \begin{center}
    \includegraphics[width=4in, height=2in]{payemsChange.png} %
	\end{center}
		\caption{NonFarm Payroll, changes}		\label{fig:payemsHard}
	\end{figure}
\end{frame}
%----------------
\begin{frame}
\frametitle{\businessProblems}
\framesubtitle{\forecast}

\textbf{What}: forecast the next month's change in non-farm payroll.

How: regress the monthly change on 
\begin{enumerate}
	\item One or more lagged values.
	\item  Unemployment rate.
	\item Civilian Labor force.
	\item Other data that might be helpful:
		\begin{itemize}
		\item GDP (quarterly)
		\item unemployment claims (weekly)
		\item ADP payroll changes (monthly)
		\item Google trend values for "job", "layoff", ... (hourly)
		\end{itemize}
\end{enumerate}

I will start with the first 3 above.
\end{frame}

%----------------
\begin{frame}
  \frametitle{\businessProblems}
  \framesubtitle{\forecast}
 	   \begin{figure}[h!]
	  \begin{center}
    \includegraphics[width=4in, height=2in]{4vars_level.png} %
	\end{center}
		\caption{target and input variables, level}		\label{fig:payemsHard}
	\end{figure}

\end{frame}

%----------------
\begin{frame}
  \frametitle{\businessProblems}
  \framesubtitle{\forecast}
 	   \begin{figure}[h!]
	  \begin{center}
    \includegraphics[width=4in, height=2in]{4vars_change.png} %
	\end{center}
		\caption{target and input variables, changes}		\label{fig:payemsHard}
	\end{figure}

\end{frame}

%----------------
\begin{frame}
  \frametitle{\businessProblems}
  \framesubtitle{\forecast}
How well does it work?  Is it \textbf{useful}?
See the true vs. predicted values below.
Traders often react to whether the number beat the expected number.  We can use 200 as a proxy for expectation, so we accurately predicted above or below about half the months.
	 \vspace{-.1in}
  \begin{figure}[h!]
	  \begin{center}
    \includegraphics[width=4in, height=1.7in]{linear.png} %
	\end{center}
		\caption{results using a linear model}		\label{fig:payemsHard}
	\end{figure}
	
	\end{frame}
		

%----------------

\subsection{\recommend}
\begin{frame}
  \frametitle{\businessProblems}
  \framesubtitle{\recommend}
  What: predict a viewer's rating for a movie.
  
  How: Embedding NN for both viewer and the movie.
  
  The similarity between viewer and rating determines the predicted rating.
  
\end{frame}


%----------------

\begin{frame}[fragile]
  \frametitle{\businessProblems}
  \framesubtitle{\recommend}
  Pytorch model
  {\scriptsize
  \begin{lstlisting}
class MF(nn.Module):
    def __init__(self, n_users, n_movies, emb_size=100):
        super(MF, self).__init__()
        self.n_users = n_users
        self.n_movies = n_movies
        self.user_emb = nn.Embedding(n_users, emb_size)
        self.movie_emb = nn.Embedding(n_movies, emb_size)
        
        # initializing the matrices with a positive number 
        # supposed to help generally will yield better results
        self.user_emb.weight.data.uniform_(0, 0.5)
        self.movie_emb.weight.data.uniform_(0, 0.5)
    
    def forward(self, users, movies):
        m = self.movie_emb(movies)
        u = self.user_emb(users)
        return (u * m).sum(1)  # taking the dot product
  \end{lstlisting}}
\end{frame}

%----------------

\subsection{\recommend}
\begin{frame}
  \frametitle{\businessProblems}
  \framesubtitle{\recommend}
 	   \begin{figure}[h!]
	  \begin{center}
    \includegraphics[width=4in, height=2in]{train_loss.png} %
	\end{center}
		\caption{Interface to gather 5 basic facts}		\label{fig:complaintInput}
	\end{figure}  
  
 	   \begin{figure}[h!]
	  \begin{center}
    \includegraphics[width=4in, height=2in]{val_loss.png} %
	\end{center}
		\caption{Interface to gather 5 basic facts}		\label{fig:complaintInput}
	\end{figure}   
  
  Test error was 1.23 with 64 dimensional embeddings
\end{frame}

%------------------------------
\subsection{\recommend}
\begin{frame}
  \frametitle{\businessProblems}
  \framesubtitle{\recommend}
  
 	   \begin{figure}[h!]
	  \begin{center}
    \includegraphics[width=4in, height=2in]{val_loss.png} %
	\end{center}
		\caption{Interface to gather 5 basic facts}		\label{fig:complaintInput}
	\end{figure}   
  
  Test error was 1.23 with 64 dimensional embeddings
\end{frame}
%----------------

\subsection{\genai}
\subsubsection{\gen}
\begin{frame}
\frametitle{\businessProblems}
  \framesubtitle{\genai}
  
    What:  Given a minimal set of \emph{facts}, 
  generate a narrative or story based on the facts.
  
\end{frame}



%----------------


\begin{frame}
\frametitle{\businessProblems}
  \framesubtitle{\genai}
 	   \begin{figure}[h!]
	  \begin{center}
    \includegraphics[width=4in, height=2in]{complaint_input.png} %
	\end{center}
		\caption{Interface to gather 5 basic facts}		\label{fig:complaintInput}
	\end{figure}  
  
  
\end{frame}


%----------------


\begin{frame}
\frametitle{\businessProblems}
  \framesubtitle{\genai}
  
 	   \begin{figure}[h!]
	  \begin{center}
    \includegraphics[width=4in, height=2in]{complaint_output_polite.png} %
	\end{center}
		\caption{resulting "polite" letter}		\label{fig:}
	\end{figure}  
  
  
\end{frame}
%----------------


\begin{frame}
\frametitle{\businessProblems}
  \framesubtitle{\genai}
 	   \begin{figure}[h!]
	  \begin{center}
    \includegraphics[width=4in, height=2in]{complaint_output_livid.png} %
	\end{center}
		\caption{resulting "livid" letter}		\label{}
	\end{figure}  
  
  
\end{frame}

%----------------

%----------------

\subsubsection{\summary}
\begin{frame}
\frametitle{\businessProblems}
  \framesubtitle{\genai}
  
  \textbf{What}: Condense a lengthy passage of several pages down to several paragraphs.
  Click the link below for a demo.
  
    \href{https://huggingface.co/spaces/jmuller/summarize}{\beamergotobutton{Link}}
    
    
  Please note a few things:
  \begin{itemize}
  \item You will probably have to restart the app
  \item Be patient, I am using limited processing 
  \end{itemize}

  
  
\end{frame}

%----------------

\section{\conclusion}
\begin{frame}
  \frametitle{\conclusion}

Let me help you solve your problems.

Next steps?
\end{frame}

%----------------
\begin{frame}
  \frametitle{\conclusion}
  \begin{center}
  Thank you!\\
\vspace{0.3in}  
  John H. Muller\\
  \vspace{0.2in}  
  jmuller.ics88@gtalumni.org \\
\vspace{0.2in}    
  +1 (617) 669-2204
  \end{center}
\end{frame}


\end{document}

